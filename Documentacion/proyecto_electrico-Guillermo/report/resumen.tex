\chapter{Resumen}

El análisis de la marcha es un estudio clínico sencillo que colabora en el diagnóstico y control de pacientes con alguna dificultad motora. El Laboratorio de Investigación en Reconocimiento de Patrones y plataformas Inteligentes (PRIS-Lab) de la Escuela de Ingeniería Eléctrica de la Universidad de Costa Rica cuenta con equipo de captura de movimiento apropiado para el estudio del movimiento humano. Desde este Laboratorio se propone la creación de una plataforma computacional que permita a profesionales de diversas disciplinas colaborar en la investigación del movimiento humano, desempeño deportivo y salud. Este trabajo representa un primer acercamiento a esta plataforma computacional, a través del desarrollo de infraestructura de software para estudio de la marcha, tema que ya se ha tratado en el laboratorio. A través de una búsqueda bibliográfica se determinaron las variables cinemáticas y espacio-temporales, así como las técnicas de análisis usadas con mayor frecuencia al estudiar la marcha. A partir de lo encontrado se desarrolló una infraestructura que permitiría a los investigadores comprobar hipótesis rápidamente y desarrollar nuevas técnicas de análisis y descubrir relaciones entre variables cinemáticas y espacio-temporales con las afectaciones motoras. 
